\documentclass[serif, ucs]{beamer}

\usetheme{Berlin}
\usecolortheme{dolphin}

\usepackage[utf8]{inputenc}
\usepackage[T2A]{fontenc}

\usepackage{amsmath,amssymb,amsthm}
\usepackage[russian]{babel}
\usepackage{mathrsfs, dsfont} % специальные шрифты, по типу \mathscr или \dsfont
\usepackage{comment} %для многострочных комментариев
\usepackage{xcolor} %для гиперссылок в тексте и их цвета
\usepackage{hyperref}
\usepackage{graphicx}
\usepackage{wrapfig}
\usepackage{lipsum}
\usepackage{multicol}
\usepackage{array}   % for \newcolumntype macro
\newcolumntype{C}{>{$}c<{$}} 
\graphicspath{/home/cowberry/Documents/10M/SPTYM/pics/}
\newtheorem{Def}{Определение}[section]
\newtheorem{Th}{Теорема}[section]
\newtheorem{Lm}{Лемма}[section] 
\newtheorem{Pb}{Задача}[section]
\newtheorem{Qu}{Признак}[section]
\newtheorem{St}{Утверждение}[section]
\newtheorem{Sl}{Следствие}[section]
\newtheorem{Zm}{Замечание}[section]
\newtheorem{Con}{Условие}[section]
\title[Крадущаяся змея, затаившееся полимино]{Задача 5. Крадущаяся змея, \newline затаившееся полимино}
\author[VII Санкт-Петербургский Турнир юных математиков]{Коченюк Анатолий}
\date{Школы 564 \\ \vspace{5mm} 26 - 31 марта 2019 года, г. Санкт-Петербург} 
\newcommand{\gr}[2]{\includegraphics[scale=#1]{../pics/#2}}
\newcommand{\p}[1]{#1^{\prime}}

\begin{document}

\begin{frame}
\titlepage
\end{frame}

\section[Введение]{Введение}

\begin{frame}{Введение}
	\begin{small}
	\begin{enumerate}
	\item Ломаную можно представить как путь в $\{(x, y)\in\mathds{R}^2\mid x\in \mathds{Z} \text{ или } y\in\mathds{Z}\}$
	Введём специальные обозначения, для задания ломаной.
	\begin{enumerate}
		\item[] $a, A$ -- отрезки направленные вправо и влево
		\item[] $b, B$ -- вверх и вниз
		
		Для удобства будут также использоваться аналоги:
		\begin{itemize}
			\item $a^{-1} = A$
			\item $b^{-1} = B$
		\end{itemize}
	\end{enumerate}
	\item Общее количество таких отрезков будем называть длиной ломаной.
	\begin{enumerate}
		\item []Допускается случай ломаной с длиной 0 --- $\varepsilon$
	\end{enumerate}
	\end{enumerate}	
	\end{small}
	
\end{frame}

\begin{frame}
\begin{enumerate}
	\item Ломаная замкнутая, если её конец совпадает с началом
	\item Ломаная простая, если у неё нет самопересечений по вершинам (допускается пересечение начала и конца -- случай замкнутой ломаной)
	\item За алфавит $S$ будем обозначать множество $S = \{a, b, a^{-1}, b^{-1}\} = \{a, b, A, B\}$ 	
\end{enumerate}

Введём 2 операции над ломаными:
\begin{enumerate}
	\item Вытягивание и затягивание -- добавление в любое место пути $l\in\{aA, Aa, bB, Bb\}$
	\item Перенос -- мы можем свободно перемещать в любое место пути определённые комбинации. 
	\end{enumerate}
\end{frame}

\begin{frame}{Дополнительные определения}
	\begin{enumerate}
		\item $l_1 \equiv l_2\Longleftrightarrow$ одну можно перевести в другую
		\item Префикс $l$ -- ломаная, идущая по тому же маршруту и не превышающая по длине
		\item Ломаная кратчайшая $\Longleftrightarrow$ нет эквивалентной с меньшей длиной
		\item Ломаная максимальная $\Longleftrightarrow$ ломаная кратчайшая и единственная кратчайшая ломаная, префиксом которой она является -- это она сама
	\end{enumerate}
\end{frame}

\begin{frame}
	\begin{footnotesize}
	Также стоит ввести такую конструкцию, как коммутатор:
	
	\begin{Def}
		Коммутатор двух букв $x, y$ из алфавита $S$  $[x, y] = xyx^{-1}y^{-1}$ 
	\end{Def}
	
	Понятно, что каждый коммутатор либо эквивалентен пустому слову, либо по второму движению коммутативен со всеми элементами.
	
	Кроме того, с его помощью можно менять местами буквы следующим образом:
	
	$[a, b] ba = aba^{-1}b^{-1}ba\equiv ab$
	
	Аналогично меняются местами другие буквы. В ячейке я буду записывать то, что нужно добавить.
	
	\begin{tabular}{C|C|C|C|C|}
		&a&b&a^{-1}&b^{-1}\\
		\hline
		a&\times &[a, b]&\times&[a, b]^{-1}\\
		\hline
		b&[a, b]^{-1}&\times&[a, b]&\times\\
		\hline
		a^{-1}&\times&[a, b]^{-1}&\times&[a, b]\\
		\hline
		b^{-1}&[a, b]&\times&[a, b]^{-1}&\times\\
	\end{tabular}
	\end{footnotesize}
\end{frame}
\section{Пункт 1}
\begin{frame}{Пункт 1 a, б}
	\begin{footnotesize}
	\begin{enumerate}
		\item $bbb \equiv babAAba\qquad$ $babA[AbaB]BB\equiv
ba[AbaB]bABB = b[aA]ba[Bb]ABB \equiv
bb[aA]BB \equiv b[bB]B \equiv bB \equiv \varepsilon \Rightarrow bbb \equiv
bbabAAba$
		\item $bbA\not\equiv aaB$. Отметим, что никакое движение не изменяет конечную точку и сумму степеней при каждой из букв (если заменить $A$ и $B$ на $a^{-1}$ и $b^{-1}$).
	\end{enumerate}
	\end{footnotesize}
	
	\gr{0.2}{bbb} \gr{0.2}{babAAba}\gr{0.2}{bbA}\gr{0.2}{aaB}
\end{frame}

\begin{frame}{Пункт 1 в}
	\begin{enumerate}
		\item []$abAB\not\equiv aabbAABB$
		\begin{Def}
			У отрезков ломаной есть направление. Будем считать, что если в многоугольнике стороны направлены против часовой стрелке, то площадь положительна. В противном случае считаем её отрицательной.
		\end{Def}
		\begin{Th}
		При действии движений, если рассматривать многоугольник составленный из точек ломаной в порядке букв в слове, то его ориентированная площадь не изменится. Такой многоугольник существует, когда ломаная замкнутая.
		\end{Th}
	\end{enumerate}
\end{frame}

\begin{frame}
	\gr{0.4}{abAB} \gr{0.4}{aabbAABB}
	
	Очевидно, что площадь первой ломаной (1) не равна площади второй (4), а значит они неэквивалентны.
\end{frame}
%%%%%%%%%%%%%%%%%%%%%%%%%%%%%%%%%%%%%%%%%%%%%%%%%%
%%%%%%%%%%%%%%%%%%%%%%%%%%%%%%%%%%%%%%%%%%%%%%%%%%

\section{Пункт 2}
\begin{frame}{Пункт 2a}
	\begin{Def}
		Ломаная является кратчайшей, если с помощью наших движений её нельзя перевести в ломаную с меньшей длиной.
	\end{Def}	
	$abAB[abAB] \equiv a[abAB]bAB\equiv aabAAB$
	
\end{frame}

\begin{frame}{Пункт 2б}
\begin{footnotesize}
\begin{Th}
	У каждой ломаной есть единственное представление в виде $[a, b]^zb^ya^x$, где $[a, b] = aba^{-1}b^{-1}$, степень --- конкатенация, т.е. $[a, b]^2 = [a, b][a, b]$, а $x, y, z\in\mathds{Z}^3$
\end{Th}

Тогда классы ломаных можно обозначать за $(z, y, x) = [[a, b]^zb^ya^x]$

\begin{Lm}
	Если у ломаной некратчайший префикс, то сама ломаная -- некратчайшая.
\end{Lm}
\begin{Def}
	Класс $(z, y, z)$ будем называть максимальным, если $len((z, y, x)*class(s))<len((z, y, x)) , \forall s\in S$
	Если в классе есть максимальная ломаная, то класс максимальный, а если класс максимальный, то все кратчайшие ломаные в этом классе -- максимальные
\end{Def}
\end{footnotesize}
\end{frame}
\begin{frame}{}
\begin{footnotesize}
\begin{Th}
	Для кратчайшей ломаной $l$ и буквы $s$ выполнятся следующее 
	
	$ls$ -- кратчайшая ломаная $\Longleftrightarrow$ в классе $[l]$ нет кратчайшего представителя, оканчивающегося на $s^{-1}$ 
\end{Th}
\begin{proof}
	\begin{itemize}
		\item[$\Rightarrow$] $ls$ -- кратчайшая. Пусть есть такой представитель $\p ls^{-1}\equiv l$
		
		$[ls] = [\p l s s^{-1}] = [\p l]$ $\quad \p ls^{-1}$ -- кратчайшая в классе $[l]\Rightarrow |l|=|\p l s^{-1}| = |\p l| + 1\Rightarrow |\p l| = |l|-1$
		
		$\p l$ -- кратчайшая в классе $[ls]\Rightarrow |\p l | = |ls| = |l| + 1$
		
		Мы получили два противоречащих равенства $\Rightarrow $ допущение неверно.
	\end{itemize}
\end{proof}
\end{footnotesize}
\end{frame}

\begin{frame}
\begin{footnotesize}
\begin{proof}
	\begin{itemize}
		\item[$\Leftarrow$] Пусть в классе $[l]$ нет кратчайшего представителя, оканчивающегося на $s^{-1}$. И пусть $ls$ -- не кратчайшая. $[ls] = [\p l], \p l$ -- кратчайшая
		
		$len([l]) - 1 \leqslant len([ls])\leqslant len([l]) + 1$. Второе неравенство выполняется по неравенству треугольника в метрике на графе Кэли группы всех слов.
		
		Первое неравенство (тоже по неравенству треугольника): $len([ls])\geqslant len(lss^{-1}) - len([s^{-1}]) = len([l]) - 1$
		
		А тогда $|l|-1 \leqslant |\p l | = len([ls])\leqslant |l| + 1$
		
		$|l| - 1\leqslant |\p l |<|ls|\leqslant |l|+1$. Между крайними числами в этом неравенстве есть 3 числа. Ни одно из средних чисел не равно $l$, а значит они "расходятся" по крайним. А значит $|l|-1 = |\p l|$

		Понятно, что $[l] = [\p l s^{-1}]$		
		
		$len([\p ls^{-1}]) = len([l]) = |l| = |\p l| + 1 = |\p l s^{-1}|$. Таким образом существует слово в классе $[w]$, которое заканчивается на $s^{-1}$??! Противоречие с условием, а значит допущение неверно.
	\end{itemize}
\end{proof}
\end{footnotesize}
\end{frame}

\begin{frame}
	\begin{Th}[Условие на максимальность класса]
		Класс ломаных максимален $\Longleftrightarrow$ в этом классе есть кратчайшие слова, оканчивающиеся на все буквы алфавита $S$
	\end{Th}
	
	\begin{Th}
		Любая кратчайшая замкнутая ломаная -- максимальная
	\end{Th}
\end{frame}

\begin{frame}{Пункт 2 c, d}
	\begin{Th}
		Любая кратчайшая ломаная -- простая
	\end{Th}
	
	Условия:
	\begin{enumerate}
		\item $L \text{--- кратчайшая}\Longleftrightarrow\nexists \p L: \p L\equiv L ~\&~ len(\p L)<len(L)$
		\item Ломаная кратчайшая, когда она является частью кратчайшей ломаной
	\end{enumerate}
\end{frame}
%%%%%%%%%%%%%%%%%%%%%%%%%%%%%%%%%%%%%%%%%%%%%%%%%%%
%%%%%%%%%%%%%%%%%%%%%%%%%%%%%%%%%%%%%%%%%%%%%%%%%%%
\section{Пункт 3}
\begin{frame}{Пункт 3}
	Замкнутые кратчайшие ломаные -- максимальные -- простые.
	
	\begin{Def}
	Полимино -- плоские геометрические фигуры, образованные путём соединения нескольких одноклеточных квадратов по их сторонам
\end{Def}

\gr{0.4}{babaBaBAAA}
\gr{0.4}{bbbaBBaBAA}
\end{frame}
\section{Благодарность}
\begin{frame}{Благодарность}
\begin{center}
{\LARGE Спасибо за внимание!}
\end{center}
\end{frame}
\end{document}