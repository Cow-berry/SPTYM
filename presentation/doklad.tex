\documentclass[12pt,a4paper]{article}
\usepackage[utf8]{inputenc}
\usepackage[T2A]{fontenc}

\usepackage{amsmath,amssymb,amsthm}
\usepackage[russian]{babel}
\usepackage{mathrsfs, dsfont} % специальные шрифты, по типу \mathscr или \dsfont
\usepackage{comment} %для многострочных комментариев
\usepackage{xcolor} %для гиперссылок в тексте и их цвета
\usepackage{hyperref}
\usepackage{graphicx}
\usepackage{wrapfig}
\usepackage{lipsum}
\usepackage{multicol}
\usepackage{array}   % for \newcolumntype macro
\usepackage[left=2cm,right=2cm,top=2cm,bottom=2cm]{geometry}	
\newcolumntype{C}{>{$}c<{$}} 
\graphicspath{/home/cowberry/Documents/10M/SPTYM/pics/}
\newtheorem{Def}{Определение}[section]
\newtheorem{Th}{Теорема}[section]
\newtheorem{Lm}{Лемма}[section] 
\newtheorem{Pb}{Задача}[section]
\newtheorem{Qu}{Признак}[section]
\newtheorem{St}{Утверждение}[section]
\newtheorem{Sl}{Следствие}[section]
\newtheorem{Zm}{Замечание}[section]
\newtheorem{Con}{Условие}[section]
\newcommand{\gr}[2]{\includegraphics[scale=#1]{../pics/#2}}
\newcommand{\p}[1]{#1^{\prime}}
\title{Доклад}
\author{Kochenyuk Anatoly}
\date{}
\begin{document}
\maketitle
Здравствуйте участники турнира и уважаемые члены жюри. Я представляю команду ЛНМО и буду докладывать задачу 5 "Крадущаяся змея, затаившееся полимино".

В задаче предлагается исследовать поведение ломанные на объединении RZ и ZR. Их предложено обозначать с помощью заглавных  строчных букв а, но я в своей работе в основном буду использовать вместо заглавных букв строчные с индексом $-1$. Множество этих букв я называю алфавитом. У каждой ломаной задана длина равная количеству букв в её обозначении.

В задаче вводятся определения замкнутых, простых ломаных. А также вводится эквивалентность с помощью двух движений: вытягивания/затягивания и переноса. Две ломаные, которые мы с помощью этиъ движений можем перевести друг в друга будем называть эквивалентными. 

Дополнительно в задаче вводятся ещё три определения: префикса ломаной и свойств кратчайшести и максимальности.

И, наконец, я в работе пользуюсь такой конструкцией, как коммутатор двух букв. Он обозначается и определяется следующим образом [показывает на "экран"]. Очевидно, что такая конструкция коммутирует со всеми буквами. Кроме того с помощью этой конструкции можно менять две соседние буквы местами следующим образом. ["экран"]. В таблице представлены коммутаторы, которые нужно "добавить", чтобы поменять местами две соседние буквы.

Теперь перейдём к самим пунктам задачи. 

Первый пункт решается в одном случае чередой применений движений, во втором замечанием про сумму степеней при каждой из букв, а в третьем замечанием про сохранение ориентированной площади у замкнутых ломаных.

Второй пункт гораздо более громоздкий. 

Её первый подпункт очевиден, а ответ отрицательным, так как можно предоставить эквивалентную ломаную с меньшей длиной. 

А перед доказательством второго необходимо доказать ещё несколько фактов. Во-первых у каждого слова есть единственное представление следующего вида ["экран"]. А раз у любой ломаной есть единственное представление, то им логично обозначать класс эквивалентных ломаных, к тому же можно упростить обозначение такого класса до тройки целых чисел. Далее введём важную лемму гласящую, что если префикс ломаной некратчайший, то и сама ломаная некратчайшая. Раз мы завели конструкцию класса, то введём понятие макимальности класса следующим образом. Таким образом, если  в классе есть максимальная ломаная, то класс максимален, а если класс макимален, то все кратчайшие в нём слова -- максимальны. 
Далее сформулируем следующую теорему: Для кратчайшей ломаной $l$ и буквы $s$ из алфавита выполняется, что кратчайшесть $ls$ эквивалентна тому, что в классе, образованном ломаной $l$ нет кратчайшей ломаной, оканчивающейся на $s^{-1}$. В письменной работе предоставлено не совсем верное доказательство, так как была отправлена работа не с последней правкой. В связи с этим я сейчас предоставлю её доказательство. 

Итак, сначала докажем необходимость: Дано, что $ls$ -- кратчайшая ломаная. Допустим, что найдётся нужный нам кратчайший представитель. Заметим, что следющие классы равны. Такое слово [показывает на $\p l s^{-1}$] -- кратчайшее в классе $l$, а и $l$ по условию кратчайшая, то их длины равны между собой и равны такой сумме. Из этого следует следющее равенство. $\p l$, как префикс кратчайшей ломаной -- кратчайшая в классе $ls$, а тогда их длины равны и равны следующей сумме. Но эти два равенства противоречат друг другу, а значит допущение неверно.

Теперь докажем достаточность: Дано, что в классе $[l]$ нет кратчайшего представителя оканчивающегося на $s^{-1}$. Допустим, что $ls$ -- некратчайшая. Обозначим за $\p l$ кратчайшй представитель класса $[ls]$. 

Заметим, что выполняется следующее неравенство: вторая часть из-за неравенства треугольника в метрике на графе Кэли группы всех слов. А первое доказывается следующим образом: применим неравенство треугольника и сделаем небольшие преобразования и вот она первая часть неравенства.

Теперь запишем это неравенство с помощью длин слов, а не классов. Так как $ls$ -- не кратчайший в классе с кратчайшим представителем $\p l$, то выполняется строгое неравенство $|\p l|<|ls|$

\end{document}