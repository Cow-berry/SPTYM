\documentclass{article}
\usepackage[utf8]{inputenc}
\usepackage{amsmath,amssymb,amsthm}
\usepackage[T2A]{fontenc}
\usepackage[russian]{babel}
\usepackage{mathrsfs, dsfont} % специальные шрифты, по типу \mathscr или \dsfont
\usepackage{comment} %для многострочных комментариев
\usepackage{xcolor} %для гиперссылок в тексте и их цвета
\usepackage{hyperref}
\usepackage{graphicx}
\usepackage{wrapfig}
\usepackage{lipsum}
\usepackage{multicol}
\graphicspath{/home/cowberry/Documents/10M/SPTYM/pics/}
\usepackage[left=2cm,right=2cm,top=2cm,bottom=2cm]{geometry}	
\author{Коченюк Анатолий}
\title{Зодача \textsuperscript{\textregistered} №5}
\newtheorem{Def}{Определение}[section]
\newtheorem{Th}{Теорема}[section]
\newtheorem{Lm}{Лемма}[section] 
\newtheorem{Pb}{Задача}[section]
\newtheorem{Qu}{Признак}[section]
\newtheorem{St}{Утверждение}[section]
\newtheorem{Sl}{Следствие}[section]
\newtheorem{Zm}{Замечание}[section]
\usepackage{relsize}
\newcommand{\vel}{\mathlarger{\mathlarger{\upsilon}}}
\newcommand{\der}[1]{\overset{\cdot}{#1}}
\newcommand{\dder}[1]{\overset{\cdot \cdot}{#1}}
\newcommand{\Lim}[2]{\lim\limits_{#1\to #2}}
\newcommand{\Ch}[1]{\overset{#1}{=}}
\newcommand{\p}[1]{#1^{\prime}}
\newcommand{\pp}[1]{#1^{\prime\prime}}
\newcommand{\ol}[1]{\overline{#1}}
\newcommand{\oll}[1]{\overline{\overline{#1}}}
\newcommand{\ov}[2]{\overset{#1}{#2}}
\newcommand{\un}[1]{\underline{#1}}
\newcommand{\gr}[2]{\includegraphics[scale=#1]{../pics/#2}}
\usepackage{comment}

\begin{document}
\maketitle

\section*{Введение}

Рассмотрим множество ломаных на решётке $\mathds{Z}\times \mathds{Z}$ с началом в точке $(0, 0)$

\begin{enumerate}
	\item Ломаную можно представить как путь в $\{(x, y)\in\mathds{R}^2\mid x\in \mathds{Z} \text{ или } y\in\mathds{Z}\}$
	Введём специальные обозначения, для задания ломаной.
	\begin{enumerate}
		\item[] $a, A$ -- отрезки направленные вправо и влево
		\item[] $b, B$ -- вверх и вниз
	\end{enumerate}
	\item Общее количество таких отрезков будем называть длиной ломаной.
	\begin{enumerate}
		\item []Допускается случай ломанной с длиной 0 -- $\varepsilon$
	\end{enumerate}
	\item Ломаная замкнутая, если её конец совпадает с началом [$(0, 0)$]
	\item Ломаная простая, если у неё нет самопересечений по вершинам (допускается перечение начала и конца -- случай замкнутой ломаной)	
\end{enumerate}

Введём 2 операции над ломанными:
\begin{enumerate}
	\item Вытягивание и затягивание -- добавление в любое место пути $l\in\{aA, Aa, bB, Bb\}$
	\item Перенос -- мы можем свободно перемещать в любое место пути определённые комбинации. В Теории групп есть понятие описывающие такие конструкции -- коммутатор.
	
	Коммутатор -- для $f, g\in G$  $ [g, h] = g^{-1}h^{-1}gh$. 
	
	Если рассмотреть группу $G = \langle a, b\rangle$ (Положим, что $A = a^{-1}$ и $B = b^{-1}$)
\end{enumerate}

Говоря о теории групп, можно задать группу всех ломанных с учётом операций:

$G =\langle a, b\mid [x, y]z = z[x, y], x, y, z\in \{a, b, A, B\}\rangle\quad\quad $ (первая операция будет автоматически учтена, т.к. в группе произведение)

\begin{Def}
	Обратное слово -- конкатенация обратных элементов в обратном порядке
\end{Def}

\begin{Zm}
	Если два слова эквивалентны, то произведение первого на обратное ко второму эквивалентно пустому слову. 
\end{Zm}
\begin{proof}
	$l\equiv m$
	
	$lm^{-1} \equiv ll^{-1}\equiv \varepsilon$
\end{proof}

\begin{Zm}
	Преобразования не изменяют  конечную точку
\end{Zm}
\begin{proof}
\begin{enumerate}
	\item мы добавляем движение в любую сторону и обратное к нему. Таким образом мы приходим в ту же точку
	\item ломаная двигается по квадрату, приходя в ту же точку
\end{enumerate}
\end{proof}
\section{Проверка на эквивалентность}

\begin{enumerate}
	\item $babAAba$ и $bbb$. $babA[AbaB]BB\equiv ba[AbaB]bABB = b[aA]ba[Bb]ABB\equiv bb[aA]BB \equiv b[bB]B \equiv bB\equiv \varepsilon\Rightarrow bbb \equiv bbabAAba$
	\item C. 
	\begin{Zm}
	что никакое движение не изменяет чётность количества букв а и b (и заглавных и строчных)
	\begin{itemize}
		\item вставка $zZ, z\in \{a, b, A, B\}$ -- добавляет две одинаковые буквы. чётность не меняется
		\item по сути это просто перемещение определённой группы букв, что не меняет их количество совсем.
	\end{itemize}
	\end{Zm}
	В этих словах разное по чётности количество букв $a$ и $b\Rightarrow$ они не эквивалентны
	\item $aabbAABB$ и $abAB$.
	 
\end{enumerate}

\begin{multicols}{2}
\gr {0.25} {babAAba} \gr{0.25} {bbb}\columnbreak

$babAAba$ и $bbb$. 
$babA[AbaB]BB\equiv ba[AbaB]bABB = b[aA]ba[Bb]ABB\equiv bb[aA]BB \equiv b[bB]B \equiv bB\equiv \varepsilon\Rightarrow bbb \equiv bbabAAba$
\end{multicols}

\begin{multicols}{2}
\gr{0.25}{bbA} \gr {0.25} {aaB}\columnbreak

$aabbAABB$ и $abAB$
\begin{Zm}
	что никакое движение не изменяет чётность количества букв а и b (и заглавных и строчных)
	\begin{itemize}
		\item вставка $zZ, z\in \{a, b, A, B\}$ -- добавляет две одинаковые буквы. чётность не меняется
		\item по сути это просто перемещение определённой группы букв, что не меняет их количество совсем.
	\end{itemize}
	\end{Zm}
	В этих словах разное по чётности количество букв $a$ и $b\Rightarrow$ они не эквивалентны
\end{multicols}

\begin{multicols}{2}
\gr{0.25}{aabbAABB} \gr{0.25}{abAB}\columnbreak

$aabbAABB$ и $abAB$

\begin{Def}
	У отрезков ломаной есть направление. Будем считать, что если в многоугольнике стороны направлены против часовой стрелке, то площадь положительна. В противном случае считаем её отрицательной.
\end{Def}
\begin{Zm}
При действии движений, если рассматривать многоугольник составленный из точек ломаной в порядке букв в слове, то его ориентированная площадь не изменится. Такой многоугольник существует, когда ломаная замкнутая.
\end{Zm}
\end{multicols}
\begin{proof}
Итак рассмотрим оба движения:
\begin{enumerate}
	\item вытягивание создаст две новых точки, но не добавит площади, т.к. на графике появится новый отрезок.
	\item вытягивание аналогично уберёт такой отрезок.
	\item перетаскивание квадрата вдоль сторон не изменит площадь, ведь площадь этого квадрата прибавляется к площади остальной фигуры, а его перемещение не меняет остальные стороны. 
\end{enumerate}

Понятно, что у слова $aabbAABB$ -- площадь 4, а у $abAB$ -- 1. А значит они не эквивалентны.
\end{proof}


\section{Кратчайшие ломаные}
\begin{Def}
	Ломаная является кратчайшим, если с помощью наших движений её нельзя перевести в ломаную с меньшей длиной.
\end{Def}

\subsection{Проверка того, что слово кратчайшее}
Является ли $abABabAB$ кратчайшей? $abAB[abAB] \equiv a[abAB]bAB\equiv aabAAB$. Мы с помощью движений получили слово меньшей длины, следовательно изначальной слово не является кратчайшим.

\subsection{Максимальные ломаные}

\begin{Def}
	Префикс ломаной L -- любая ломаная, не превосходящая по длине ломаную L и идущая по тому же маршруту. 
\end{Def}

\begin{Def}
	Ломаная называется максимальной, если она является кратчайшей и единственная кратчайшая ломаная для которой она является префиксом -- она сама.
\end{Def}

\begin{Th}
	Любая кратчайшая замкнутая ломаная -- максимальная.
\end{Th}
\begin{proof}
	фиксируем кратчайшую замкнутую ломаную  L.
	
	
\end{proof}

\end{document}